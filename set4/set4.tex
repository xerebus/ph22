\documentclass{article}
\input{../../preamble.tex}

\usepackage{fancyhdr}
\usepackage{cancel}
\usepackage[margin=1in, headheight=50pt]{geometry}
\pagestyle{fancy}
\lhead{\textbf{Ph22 Set 4}}
\chead{}
\rhead{Aritra Biswas}
\setlength{\headsep}{20pt}

\begin{document}

\section{Simulation of $N$-body systems}

\subsection{Simulation framework}

We take an object-oriented approach to simulating $N$ bodies under a mutual
gravitational attraction, although the approach is more general. A
\texttt{Particle} class holds a 3-dimensional vector position and velocity,
a mass, and a radius. Methods are included to set a particle's position
and velocity in both Cartesian and spherical coordinates; the latter is
particularly useful for generating large numbers of particles distributed
in a sphere.

\texttt{Particle} objects are contained within a \texttt{Cluster} object
which is responsible for the interaction between \texttt{Particle} objects.
\texttt{Cluster} objects also contain a reference to a force function, which
takes two \texttt{Particle} objects as arguments and returns the force vector
on the first \texttt{Particle}. In this way, the force mechanism is easily
expandable: for example, we could add a charge attribute to the
\texttt{Particle} class and define an \texttt{elec_force(p1, p2)}
function which uses \texttt{p1.charge} and \texttt{p2.charge}. Then,
the \texttt{Cluster} could easily include both gravitational and electric
forces simply by using \texttt{Cluster.force = grav_force + elec_force}.

The \texttt{Cluster} object contains an \texttt{evolve(dt)} method, which
brute-force calculates the force on every object and evolves their motion
for a timestep \texttt{dt}. The \texttt{Cluster} can also run a visualization
of its state using \texttt{vpython}, and can save its state to a file
using \texttt{cPickle}.

\subsection{Higher-order symplectic Euler method}

In symplectic Euler methods with two dynamical variables, the current value
of one variable is used to update the other, and then the updated value
is used to update the former. Thus, we have two possible sympletic Euler
methods in our case:
\begin{align}
    \vec r_{i+1} = \vec r_i + \vec v_i \, dt,
    \sp{}
    \vec v_{i+1} = \vec v_i + \vec a(\vec r_{i+1}) \, dt.
    \\
    \vec r_{i+1} = \vec r_i + \vec v_{i + 1} \, dt,
    \sp{}
    \vec v_{i+1} = \vec v_i + \vec a(\vec r_i) \, dt.
\end{align}
The behavior of these numerical integration methods was studied on the
three-body choreographic orbit (Ph22.3) using visualization. It was observed
that method (1) resulted in a collapsing orbit, while method (2) resulted
in an expanding orbit. Inspired by these results, the \texttt{Cluster}
method employs a higher-order method which alternates the two symplectic
steps, resulting in a stable three-body choreographic orbit. A visualization
can be seen using \texttt{vs_sims.py}.

\subsection{Force softening}

Since the gravitational force is inversely proportional to
$R^2$ where $\vec R \equiv \vec r - \vec r'$ is the separation vector,
extremely large forces are calculated when the particles come very close
to each other. In reality, gravity would not be the dominant effect at these
close distances, as collision and absorption would have to be considered in
our model. However, in our simplistic model, it is sufficient to prevent the
simulation from stalling at these distances.

We add a force softening parameter $a$:
\begin{align}
    \vec {F'} = {G m_1 m_2 \vec R \over (R + a)^3},
    \tsp{which under $a \to 0$ reduces to}
    \vec F = {G m_1 m_2 \vec{\hat R} \over R^2}.
\end{align}
This ensures that a finite ``fake force'' is calculated even when the
particles occupy the same position. We choose $a = 0.1$, $G = 1$ for our
simulations.


\section{Spherical distributions}

Long-timescale simulations were run on two initial conditions: $N = 1000$
particles distributed in an unit sphere, (1) with unit speeds in random
directions, and (2) with the particles initially at rest.
Figures \ref{rand.pdf} and \ref{rest.pdf} show radial position
distributions of the particles at large timesteps.

\plop{rand.pdf}{Radial position distribution of particles, with particles
    initially distributed with unit velocities in random directions in
an unit sphere and evolved for 7000 timesteps.
The formation of a core in $r < 1000$ and a halo near
$r = 3000$ is apparent.}
\plop{rest.pdf}{Radial position distribution of particles, with particles
    at rest initially distributed uniformly in
an unit sphere and evolved for 7000 timesteps.
The formation of a core in $r < 1000$ is still visible, as well as a
smaller halo at $r = 1500$ and a wider halo at
$r = 3000$.}


\end{document}
