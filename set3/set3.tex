\documentclass{article}
\input{../../preamble.tex}

\usepackage{fancyhdr}
\usepackage{cancel}
\usepackage[margin=1in, headheight=50pt]{geometry}
\pagestyle{fancy}
\lhead{\textbf{Ph22 Set 3}}
\chead{}
\rhead{Aritra Biswas}
\setlength{\headsep}{20pt}

\begin{document}


\section{Restricted three-body problem}

We study an asteroid of mass $m$ in the Jupiter-Sun system (masses $M_1$ and
$M_2$ respectively). Since $m \ll M_1, M_2$, we approximate Jupiter and the
Sun as fixed in the corotating frame, which rotates with angular velocity
$\Omega$:
\begin{align}
    \Omega = \sqrt{G(M_1 + M_2) \over R^3},
    \tsp{where $R_3$ is the Jupiter-Sun separation,}
\end{align}
with respect to an inertial frame. In this frame, Jupiter and the Sun are
fixed at:
\begin{align}
    \vec{r_1} = \left( {M_2 R \over M_1 + M_2}, 0 \right),
    \sp{}
    \vec{r_2} = \left( -{M_1 R \over M_1 + M_2}, 0 \right).
\end{align}
The acceleration of the asteroid at $\vec r = (x, y)$ is given by:
\begin{align}
    \vec{\ddot r} =
    - {GM_1} {\vec r - \vec{r_1} \over |\vec r - \vec{r_1}|^3}
    - {GM_2} {\vec r - \vec{r_2} \over |\vec r - \vec{r_2}|^3}
    + 2\Omega (\dot y, -\dot x)
    + \Omega^2 \vec r,
\end{align}
with gravitational contributions from Jupiter and the Sun, a Coriolis force
contribution, and a centrifugal force contribution. Lagrange's solution
identifies two \emph{stable} Lagrange points $\vec{L_4}$ and $\vec{L_5}$
located at:
\begin{align}
    \vec{L_{4,5}}
    = R \left({M_2 - M_1 \over M_1 + M_2} \cos\alpha_{4,5},
    \sin\alpha_{4,5}\right)
    \sp{where}
    \alpha_{4,5} = \pm {\pi \over 3}.
\end{align}
To observe the stability of these starting points, we will start our asteroid
at nearby points $\vec{\tilde L_{4,5}}$ with $\tilde \alpha_{4, 5}
\equiv \alpha_{4,5} \pm \Delta\alpha$. For small $\Delta\alpha$, we should
see small oscillations around $\vec{L_{4,5}}$, but for large $\Delta\alpha$,
we should see more variation in the asteroid's orbit. We use an RK4
integration routine to evolve the position $\vec r$ of the asteroid over
several periods $T = 2\pi / \Omega$.

Figures \ref{da_0p01.pdf} through \ref{da_1p5.pdf} show asteroid orbits for
$\Delta\alpha = 0.01, 0.05, 0.5, 1.5$. No orbit is shown for
$\Delta\alpha = 0$ because there is none: the asteroid remains stationary
at its starting position. As expected, the asteroid's orbit amplitude
increases dramatically with $\Delta\alpha$.

\plop{da_0p01.pdf}{Small-oscillation orbits of an asteroid in the Jupiter-Sun
    system, placed at starting positions $\vec{\tilde L_{4,5}}$
    with $\Delta\alpha = 0.01$. Yellow star denotes the Sun, and red circle
    denotes Jupiter.}

\plop{da_0p05.pdf}{Small-oscillation orbits of an asteroid in the Jupiter-Sun
    system, placed at starting positions $\vec{\tilde L_{4,5}}$
    with $\Delta\alpha = 0.05$. Yellow star denotes the Sun, and red circle
    denotes Jupiter.}

\plop{da_0p5.pdf}{Larger-oscillation orbits of an asteroid in the Jupiter-Sun
    system, placed at starting positions $\vec{\tilde L_{4,5}}$
    with $\Delta\alpha = 0.5$. Yellow star denotes the Sun, and red circle
    denotes Jupiter.}
 
\plop{da_1p5.pdf}{Larger-oscillation orbits of an asteroid in the Jupiter-Sun
    system, placed at starting positions $\vec{\tilde L_{4,5}}$
    with $\Delta\alpha = 1.5$. Yellow star denotes the Sun, and red circle
    denotes Jupiter.}

\section{General three-body problem with equal masses}

In a general multi-body problem, the acceleration of the $i$-th object is
given by:
\begin{align}
    \vec{\ddot r_i} = -G \sum_{j \not= i} M_j {\vec r_i - \vec r_j
    \over |\vec r_i - \vec r_j|^3}.
\end{align}
We consider a case with three bodies of equal mass. We will use natural
units, where $G = M_{1, 2, 3} = 1$.

\subsection{Lagrange solution}

We want to verify the Lagrange solution, where
the masses move along the vertices of a rotating equilateral triangle of
side length $d$ and velocity $v$. First, we will confirm this solution
analytically.

Let $R \equiv d / \sqrt{3}$, the distance of a point on the equilateral
triangle from the center. This is the radius of the centripetal motion of
each mass. Without loss of generality, let one of the masses begin at
$(x, y) = (0, R)$. By symmetry, the net gravitational force on this mass
is in the $y$-direction (towards the center) and its magnitude is:
\begin{align}
    F = 2 \left(G M^2 \over d^2\right)\cos 30^\circ = 
    {GM^2 \over d^2} \sqrt 3 = {GM^2 \over dR}.
\end{align}
The velocity of centripetal motion is related to the centripetal force by
$F = v^2 / R$, so:
\begin{align}
    v = \sqrt{RF} = \sqrt{GM \over d}.\,\square
\end{align}
We simulate this orbit using our RK4 integrator. Initial conditions are:
\begin{align}
    (x_i, y_i) = (R\sin\theta_i, R\cos\theta_i),
    \sp{} (\dot x_i, \dot y_i) = (v\cos\theta_i, -v\sin\theta_i),
    \sp{with} \{\theta_1, \theta_2, \theta_3\} =
    \left\{0, {2\pi \over 3}, {4\pi \over 3} \right\}.
\end{align}
The result of the simulation is saved in \texttt{lagrange.mp4} in this
directory.

As with all numerical simulations, our initial conditions are approximations
to the exact stable parameters. Figure \ref{lag_long.pdf} shows the
long-term evolution of the system: the bodies eventually leave their
stable orbit and drift outward.

\plop{lag_long.pdf}{Long-term evolution of the Lagrange solution. The
    three bodies follow the stable circle for a long time, but eventually
    drift outwards, as seen in the blue path. The red and green bodies
    meet during their outward spirals and slingshot off of each other,
shooting off to infinity.}

\subsection{Choreographic orbit}

Using the same RK4 routine, we use the given conditions for a
choreographic orbit. The simulation result is saved in
\texttt{choreographic.mp4} in this directory.



\end{document}
